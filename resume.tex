%-------------------------------------
% Resume in LaTeX
% Author: Cesar Arguello
% 
% Project: 
% Base on: https://github.com/xyz-yuanhf/yuan-resume
% License: MIT
%------------------------------------

\documentclass[a4paper, 10pt]{article}
\usepackage{myresume}  % Style package
\usepackage{hyperref}
\pagenumbering{gobble}

% --------------------  START  --------------------
\begin{document}

% -------------------- HEADING --------------------
\begin{flushright}
  \setstretch{0.6}
  \item {\Calluna cesar.n.arguello.martinez.gr@dartmouth.edu}  % E-mail address
  \item {\Calluna \url{https://www.carguellom.com}}  % Home page
\end{flushright}\vspace{-45pt}

\begin{flushleft}  % Rplace here with your name (and identity if required)
  {\Calluna \fontsize{30pt}{30pt}\selectfont \textsc{Cesar Arguello}} %\quad {\Calluna \fontsize{14.5pt}{14.5pt}\selectfont \textsc{Ph.D.}}%
  \noindent\rule{\textwidth}{0.4pt}
\end{flushleft}

% -------------------- EDUCATION --------------------
\sectionBlock{
\section{Education}
}{
\eduHeading
  {Ph.D. in Computer Science}{September 2022-June 2027\footnotesize{\textit{(expected)}}}
  {Dartmouth College}{Hanover, NH, USA}
\itemListStart
  \myItem{Advisor: Dr. David Kotz}
  \myItem{Research area: Security and Privacy in the Lifecycle of IoT for Consumer Environments}
\itemListEnd

\eduHeading
  {B.S. in Physics \& Computer Science}{August 2018-May 2022}
  {University of Florida}{Gainesville, FL, USA}
\itemListStart
  \myItem{Graduted Cum Laude}
  \myItem{Selected for Honors Program}
\itemListEnd

}




% -------------------- INDUSTRY EXPERIENCE --------------------
\sectionBlock{
\section{Industry Experience}
}{
\eduHeading
  {Product Development Engineer Intern}{January 2022-May 2022 }
  {Intel Corporation}{Santa Clara, CA, USA}
\itemListStart
  \myItem{Designed, developed, and debugged sort test programs for server products.}
  \myItem{Tested, validated, modified, and redesigned circuits to guarantee component margin to
  specification.}
  \myItem{Analyzed and evaluated component specification versus performance to ensure optimal
  match of component requirements with production equipment capability.}
\itemListEnd
}

% -------------------- RESEARCH EXPERIENCE --------------------
\sectionBlock{
\section{Research Experience}
}{
  \eduHeading
  {Graduate Research Assistant}{January 2023-Present}
  {Dartmouth College}{Hanover, NH, USA}
  \itemListStart
  \myItem{Maintain a C, C++, and Python codebase for a harmonic radar controller, ensuring optimal performance and reliability for various research applications.}
  \myItem{Design and implement experiments and circuit prototypes for novel applications of harmonic radars in IoT security, focusing on advancing research and practical use cases.}
  \myItem{Develop comprehensive research plans targeting conference and journal publications while actively reviewing and discussing relevant literature on IoT security.}
  \itemListEnd

\eduHeading
{Undergraduate Research Assistant}{June 2021-May 2022}
{University of Florida}{Gainesville, FL, USA}
\itemListStart
\myItem{Designed and developed experiments to test theoretical frameworks on EM side channel disassembly.}
\myItem{Collected, processed, and analyzed EM traces using statistical and machine learning models.}
\myItem{Developed research plans leading to the successful completion of project deliverables (publications, presentations, etc.).}
\itemListEnd
}

\sectionBlock{
\section{Publications}
}{
\pubListStart
\justifying
\item \textbf{Cesar Arguello}, Beatrice Perez, Timothy J. Pierson, and David Kotz. Detecting Battery Cells with Harmonic Radar. \textit{Proceedings of the ACM Conference on Security and Privacy in Wireless and Mobile Networks (WiSec)}, 2024.
\item Timothy J. Pierson, \textbf{Cesar Arguello}, Beatrice Perez, Wondimu Zegeye, Kevin Kornegay, Carl Gunter, and David Kotz. We need a “building inspector for IoT” when smart homes are sold. \textit{IEEE Security \& Privacy}, 2024.
\item Beatrice Perez, \textbf{Cesar Arguello}, Timothy J. Pierson, Gregory Mazzaro, and David Kotz.\textit{Proceedings of the IEEE Military Communications Conference (MILCOM)}, 2023.
\item \textbf{Cesar Arguello}, Hunter Searle, Sara Rampazzi, Kevin Butler. [Poster]: A practical methodology for ML-Based EM Side Channel Disassemblers. \textit{Proceedings of the 2022 Poster Session of the 7th IEEE European Symposium on Security and Privacy}, 2022.
\pubListEnd
}

%---------------- TEACHING EXPERIENCE -------------------------\begin{itemize}
\sectionBlock{
  \section{Teaching Experience}
}{
\teachHeading
    {Graduate Teaching Assistant}{September 2022-January 2023}
    {Dartmouth College}{Hanover, NH, USA}
    {CS50 - Software Design \& Implementation}
    \itemListStart
    \myItem{Developed and maintained automated grading scripts using Bash and Python to evaluate student assignments.}
    \myItem{Conducted in-depth code reviews for programming projects, providing constructive feedback to enhance code quality, efficiency, and adherence to best practices.}
    \myItem{Held office hours to provide one-on-one mentorship to students on course material and programming projects.}
    \itemListEnd
\teachHeading
    {Undergraduate Teaching Assistant}{January 2021-May 2021}
    {University of Florida}{Gainesville, FL, USA}
    {CDA3101 - Introduction to Computer Organization}
    \itemListStart
    \myItem{Assisted student achieved proposed academic goals by leading discussions on lecture material.}
    \myItem{Supported instructors by managing the evaluation of course assignments and providing detailed feedback to students.}
    \myItem{Held office hours to review course material, answer general questions, and provide some assistance on assignments.}
    \itemListEnd
}

% -------------------- PUBLICATIONS --------------------


% -------------------- AWARDS & HONORS --------------------
% \sectionBlock{
% \section{Awards\\and\\Honors}
% }{
% \awardListStart
% \justifying
% \item \emph{Academic Scholarship}, Davis UWC Scholars \hfill August 2018
% \item \emph{Honor Society}, Phi Betta Kappa \hfill May 2022
% \awardListEnd
% }

% -------------------- SKILLS --------------------
\sectionBlock{
\section{Skills}
}{
\skillListStart
\justifying
\item \emph{Languages}: English, Spanish.
\item \emph{Programming}: {\Courier \fontsize{11pt}{11pt}\selectfont Python}, {\Courier \fontsize{11pt}{11pt}\selectfont C}, {\Courier \fontsize{11pt}{11pt}\selectfont C++},  {\Courier \fontsize{11pt}{11pt}\selectfont LaTex}, {\Courier \fontsize{11pt}{11pt}\selectfont x86}
\item \emph{Frameworks}: {\Courier \fontsize{11pt}{11pt}\selectfont PyTorch}
\item \emph{Software}: Microsoft Suite, Git, GDB, Wireshark
\item \emph{Web Skills}: HTML5, CSS
\skillListEnd
}


\end{document}

